\pagebreak

\DefineVerbatimEnvironment{Highlighting}{Verbatim}{commandchars=\\\{\},fontsize=\small}

\section{Foreword}\label{foreword}

This course is aimed at beginner-to-intermediate Scala developers who
want to get started using the {[}Play 2{]} web framework. By the end of
the course we will have a solid foundation in each of the main libraries
Play provides for building sites and services:

\begin{itemize}
\itemsep1pt\parskip0pt\parsep0pt
\item
  Routing, controllers, and actions
\item
  Manipulating requests and responses
\item
  Generating HTML
\item
  Parsing and validating form data
\item
  Reading and writing JSON
\item
  Asynchronous request handling
\item
  Calling external web services
\end{itemize}

As coursework we will build a simple chat application from the ground
up. We will start with a very basic web site and end up building a
complete service-oriented architecture with each concern separated out
to a separate microservice.

The material presented focuses on Play version 2.3, although the
approaches introduced are generally applicable to Play 2.2+.
